\newcommand*{\headerstyle}[1]{{\fontsize{20pt}{1em}\headerfontlight\color{MaterialGrey800} #1}}
\newcommand*{\secondaryheaderstyle}[1]{{\normalsize\headerfont\color{black} #1}}
\newcommand*{\footerstyle}[1]{{\fontsize{8pt}{1em}\footerfont\scshape\color{MaterialGrey400} #1}}

\newcommand*{\sectionstyle}[1]{{\fontsize{11pt}{1em}\bodyfont\bfseries\color{MaterialBlueGrey900} #1}}
\newcommand*{\entrytitlestyle}[1]{{\fontsize{7pt}{1em}\bodyfont\scshape\color{MaterialBlueGrey800} #1}}
\newcommand*{\entrylocationstyle}[1]{{\fontsize{7pt}{1em}\bodyfont #1}}

% macro for section title
\newcommand{\customsection}[1]{%
  \vspace{-\baselineskip}
  \adjustbox{margin=0 .1cm 0 0}{
    \sectionstyle{%
      % \resizebox{\dimexpr\linewidth-2\baselineskip\relax}{!}{%
        % draw a horizontal line with a specific thickness
        \begin{tikzpicture}[]
          \fill[MaterialLightBlue200, anchor=north west] (0, -1.8ex) rectangle (\linewidth, -1.8ex + -0.18cm);
          \node[anchor=north west] (txt) at (0, 0) {#1};
        \end{tikzpicture}
      % }
    }
    \phantomsection
  }\\
}

% macro for list entry
\newcommand{\entry}[5]{%
  \begin{minipage}{\linewidth}
    \textbf{#1}\hfill#2
    \\[-0.5ex]
    \ifx&#3&
    \else
      \entrytitlestyle{\MakeUppercase{#3}}\hfill\entrylocationstyle{#4}\par
    \fi
    % \ifx&#5&% is #5 empty?
    %   \vspace{1.25ex}
    % \else
    %   #5
    % \fi
    #5
    \vspace{1.1ex}
  \end{minipage}
}
